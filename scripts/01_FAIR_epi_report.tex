% Options for packages loaded elsewhere
\PassOptionsToPackage{unicode}{hyperref}
\PassOptionsToPackage{hyphens}{url}
%
\documentclass[
  10pt,
]{article}
\usepackage{amsmath,amssymb}
\usepackage{lmodern}
\usepackage{iftex}
\ifPDFTeX
  \usepackage[T1]{fontenc}
  \usepackage[utf8]{inputenc}
  \usepackage{textcomp} % provide euro and other symbols
\else % if luatex or xetex
  \usepackage{unicode-math}
  \defaultfontfeatures{Scale=MatchLowercase}
  \defaultfontfeatures[\rmfamily]{Ligatures=TeX,Scale=1}
\fi
% Use upquote if available, for straight quotes in verbatim environments
\IfFileExists{upquote.sty}{\usepackage{upquote}}{}
\IfFileExists{microtype.sty}{% use microtype if available
  \usepackage[]{microtype}
  \UseMicrotypeSet[protrusion]{basicmath} % disable protrusion for tt fonts
}{}
\makeatletter
\@ifundefined{KOMAClassName}{% if non-KOMA class
  \IfFileExists{parskip.sty}{%
    \usepackage{parskip}
  }{% else
    \setlength{\parindent}{0pt}
    \setlength{\parskip}{6pt plus 2pt minus 1pt}}
}{% if KOMA class
  \KOMAoptions{parskip=half}}
\makeatother
\usepackage{xcolor}
\IfFileExists{xurl.sty}{\usepackage{xurl}}{} % add URL line breaks if available
\IfFileExists{bookmark.sty}{\usepackage{bookmark}}{\usepackage{hyperref}}
\hypersetup{
  pdftitle={Digital epigraphy in 2022: state of the art},
  pdfauthor={Petra Hermankova; Marietta Horster; Jonathan Prag},
  pdfkeywords={inscriptions, digital-epigraphy, FAIR data
principles, Linked Open Data, survey},
  hidelinks,
  pdfcreator={LaTeX via pandoc}}
\urlstyle{same} % disable monospaced font for URLs
\usepackage[margin=1in]{geometry}
\usepackage{longtable,booktabs,array}
\usepackage{calc} % for calculating minipage widths
% Correct order of tables after \paragraph or \subparagraph
\usepackage{etoolbox}
\makeatletter
\patchcmd\longtable{\par}{\if@noskipsec\mbox{}\fi\par}{}{}
\makeatother
% Allow footnotes in longtable head/foot
\IfFileExists{footnotehyper.sty}{\usepackage{footnotehyper}}{\usepackage{footnote}}
\makesavenoteenv{longtable}
\usepackage{graphicx}
\makeatletter
\def\maxwidth{\ifdim\Gin@nat@width>\linewidth\linewidth\else\Gin@nat@width\fi}
\def\maxheight{\ifdim\Gin@nat@height>\textheight\textheight\else\Gin@nat@height\fi}
\makeatother
% Scale images if necessary, so that they will not overflow the page
% margins by default, and it is still possible to overwrite the defaults
% using explicit options in \includegraphics[width, height, ...]{}
\setkeys{Gin}{width=\maxwidth,height=\maxheight,keepaspectratio}
% Set default figure placement to htbp
\makeatletter
\def\fps@figure{htbp}
\makeatother
\setlength{\emergencystretch}{3em} % prevent overfull lines
\providecommand{\tightlist}{%
  \setlength{\itemsep}{0pt}\setlength{\parskip}{0pt}}
\setcounter{secnumdepth}{5}
\newlength{\cslhangindent}
\setlength{\cslhangindent}{1.5em}
\newlength{\csllabelwidth}
\setlength{\csllabelwidth}{3em}
\newlength{\cslentryspacingunit} % times entry-spacing
\setlength{\cslentryspacingunit}{\parskip}
\newenvironment{CSLReferences}[2] % #1 hanging-ident, #2 entry spacing
 {% don't indent paragraphs
  \setlength{\parindent}{0pt}
  % turn on hanging indent if param 1 is 1
  \ifodd #1
  \let\oldpar\par
  \def\par{\hangindent=\cslhangindent\oldpar}
  \fi
  % set entry spacing
  \setlength{\parskip}{#2\cslentryspacingunit}
 }%
 {}
\usepackage{calc}
\newcommand{\CSLBlock}[1]{#1\hfill\break}
\newcommand{\CSLLeftMargin}[1]{\parbox[t]{\csllabelwidth}{#1}}
\newcommand{\CSLRightInline}[1]{\parbox[t]{\linewidth - \csllabelwidth}{#1}\break}
\newcommand{\CSLIndent}[1]{\hspace{\cslhangindent}#1}
\usepackage{booktabs}
\usepackage{longtable}
\usepackage[bf,singlelinecheck=off]{caption}
\usepackage[scale=.8]{sourcecodepro}

\usepackage{framed,color}
\definecolor{shadecolor}{RGB}{248,248,248}


    \makeatletter
    \let\@fnsymbol\@alph
    \makeatother
\usepackage{graphicx} \usepackage{fancyhdr} \pagestyle{fancy} \setlength\headheight{18pt} \fancyhead[R]{\includegraphics[width=1.2cm]{../assets/logo.jpg}}
\ifLuaTeX
  \usepackage{selnolig}  % disable illegal ligatures
\fi

\title{Digital epigraphy in 2022: state of the art\thanks{The FAIR
Epigraphy project was funded by \ldots{}}}
\author{Petra Hermankova\footnote{Johannes Gutenberg University in
  Mainz,
  \href{mailto:petra.hermankova@uni.mainz.de}{\nolinkurl{petra.hermankova@uni.mainz.de}},
  \url{https://orcid.org/0000-0002-6349-0540}} \and Marietta
Horster\footnote{Johannes Gutenberg University in Mainz,
  \href{mailto:horster@uni.mainz.de}{\nolinkurl{horster@uni.mainz.de}},
  \url{https://orcid.org/0000-0003-1434-224X}} \and Jonathan
Prag\footnote{Oxford University,
  \href{mailto:jonathan.prag@merton.ox.ac.uk}{\nolinkurl{jonathan.prag@merton.ox.ac.uk}},
  \url{https://orcid.org/0000-0003-3819-8537}}}
\date{25 April, 2022}

\begin{document}
\maketitle
\begin{abstract}
This document maps the state of digital epigraphy in 2022, with a focus
on Open Science practices and accessibility of resources. The report is
based on responses received during the digital survey circulating
between February and April 2022, organised by the FAIR Epigraphy
Project. The responses cover a broad spectrum of projects from Europe
and the US, ranging from well established projects with a relatively
stable institutional support to a short-term projects with more narrow
focus and limited access to IT support and funding. The results of the
survey will be used to inform the planning of the FAIR Epigraphy project
in the following three years.
\end{abstract}

\hypertarget{introduction}{%
\section{Introduction}\label{introduction}}

The field of digital epigraphy has seen significant development in
recent years: not only are traditional epigraphic corpora increasingly
being digitised and made accessible via their websites for anyone to
browse and search but several resources are already born digital without
any printed edition, e.g., \emph{Inscriptions of Greek Cyrenaica}
(\protect\hyperlink{ref-roueche_inscriptions_2020}{Roueche \emph{et
al.}, 2020}), \emph{Inscriptions of Roman Tripolitania}
(\protect\hyperlink{ref-roueche_inscriptions_2022}{Roueche, 2022}); for
more see (\protect\hyperlink{ref-bruun_epigraphy_2015}{Elliott, 2015}).
Most inscriptions contain references to places, people or events, or
contain spatio-temporal data related to the place and time of their
creation and provide an ideal resource to study past communities as a
whole. However, in order to be able to harness their full potential and
for example access \emph{all} inscriptions from a place of interest or
of a given type, we need to link the existing datasets together. The
concept of \emph{Linked Open Data} (LOD) provides a means of connecting
various digital datasets while enriching the text with broader
spatio-temporal context as well as prosopographic data, leading to the
creation of new connections between individual inscriptions as well as
archaeological sites or potential re-evaluation of historical narratives
(\protect\hyperlink{ref-bagnall_pleiades_2006}{Bagnall \emph{et al.},
2006}; \protect\hyperlink{ref-geser_wp15_2016}{Geser, 2016};
\protect\hyperlink{ref-tupman_where_2021}{Tupman, 2021}). Although many
epigraphic datasets have been using LOD, especially to record the
spatial component by using Pleiades or Trismegistos, there is still a
considerable gap in the LOD implementation across the discipline and
thus the accessibility of the data.

The contribution of individual projects can be beneficial to groups
sharing similar interests (i.e., geographic area, chronological period,
linguistic environment) but is rather limited to the epigraphic
discipline as a whole. The value of LOD lies in being able to build on
the efforts and investment of numerous generations of epigraphers who
relentlessly produced high-quality publications in an analogue and
nowadays, in a digital form. Whether there is one master database
connecting all the inscriptions to one, or not, once the data is FAIR
and linked to other LOD, new avenues of research open - either to large
scale comparative studies such as
(\protect\hyperlink{ref-assael_restoring_2022}{Assael \emph{et al.},
2022}; \protect\hyperlink{ref-hermankova_inscriptions_2021}{Heřmánková
\emph{et al.}, 2021}) or projects working on the same material but with
different emphases (\protect\hyperlink{ref-mullen_manual_2021}{Mullen \&
Bowman, 2021}; \protect\hyperlink{ref-willi_manual_2021}{Willi, 2021}).
Once the data are linked, there is no need to build one central
repository, which is often costly and non-sustainable in the long run as
documented by the recent experience of the EAGLE Portal
(\protect\hyperlink{ref-orlandi_digital_2021}{Orlandi, 2021}), but
rather to empower individual users and provide them with clear
guidelines and skills on how to work with LOD in epigraphy.

The \textbf{FAIR Epigraphy Project}
(\url{https://www.csad.ox.ac.uk/fair-epigraphy}) aims to fill in the gap
between the digitisation of inscriptions and being able to use their
full potential as a digital resource. The FAIR Epigraphy project has
been established as a collaboration between Johannes Gutenberg
University in Mainz (Prof.~Marietta Horster) and the University of
Oxford (Prof.~Jonathan Prag), funded by the Arts and Humanities Research
Council (AHRC) and Deutsche Forschungemeinschaft (DFG) and will run for
36 months from 2022 to 2025. \textbf{FAIR Epigraphy} aims to create an
interactive platform for all epigraphic projects, aligning their digital
needs with the principles of FAIR science. The overall desirability for
\textbf{FAIR} - \emph{Findable}, \emph{Accessible},
\emph{Interoperable}, \emph{Reusable}
(\protect\hyperlink{ref-wilkinson_fair_2016}{Wilkinson \emph{et al.},
2016}) data is fundamental advancing research into the epigraphic,
linguistic, and material culture of the ancient world.

\begin{quote}
\emph{``The principles emphasise machine-actionability (i.e., the
capacity of computational systems to find, access, interoperate, and
reuse data with none or minimal human intervention) because humans
increasingly rely on computational support to deal with data as a result
of the increase in volume, complexity, and creation speed of data.''}
(FAIR Principles website,
\url{https://www.go-fair.org/fair-principles/})
\end{quote}

With the increase in Linked Open Data and novel interface technologies
and standards, the FAIR Epigraphy project will be able to create the
tools and the community needed to transform epigraphic research in the
digital age. However, the FAIR Epigraphy project does not wish to
replicate any current efforts, but rather to align existing initiatives
and bring them together to create a hub of high-quality tools and FAIR
compliant standards and resources for the modern epigraphic discipline.
Our internationally collaborative approach will enable and support
innovative research across epigraphic data, and the wider linked web of
data (especially archaeological data), such that all epigraphic data is
increasingly FAIR for both the research community and the wider public.
To that end, we aim to:

\begin{enumerate}
\def\labelenumi{\arabic{enumi}.}
\tightlist
\item
  consolidate community-wide standards (vocabularies and ontology);
\item
  host and make fully accessible the resulting linked open data
  published by individual projects (RDF/XML data publication);
\item
  develop the tools for community implementation of those standards
  (vocabulary and ontology hosting and publication);
\item
  provide support to members of the community in implementing the
  standards within existing and new projects.
\end{enumerate}

In order to map the existing field of digital epigraphy, current
practices and standards, as well as clarify the (digital) needs of the
discipline, we have circulated the two scoping surveys in February 2022
(\href{https://github.com/FAIR-epigraphy/scoping_survey_report/data/01_Survey_partners_questions.pdf}{\emph{FAIR
Epigraphy: Scoping survey for partners and collaborators}} and
\href{https://github.com/FAIR-epigraphy/scoping_survey_report/data/02_Survey_projects_questions.pdf}{\emph{Digital
epigraphy in 2022: scoping survey}} for all digital epigraphy projects).
The results of the surveys, presented in the current report, will be
used to plan the activities and efficiently allocate the resources of
the FAIR Epigraphy Project in the next three years. The survey answers
are anonymised so that individual projects cannot be identified on the
basis of their replies and the data is stored as a TSV (tab-separated
value) file within the project's GitHub repository
(\url{https://github.com/FAIR-epigraphy/scoping_survey_report/}) as a
supplement to the text of this report and can be accessed under the
CC-BY-SA 4.0 International License.

\hypertarget{fair-epigraphy-partner-projects}{%
\section{FAIR Epigraphy partner
projects}\label{fair-epigraphy-partner-projects}}

This section summarises the results of the online survey
\href{https://github.com/FAIR-epigraphy/scoping_survey_report/data/01_Survey_partners_questions.pdf}{\emph{FAIR
Epigraphy: Scoping survey for partners and collaborators}} aimed at the
established digital projects that are already official partners and
collaborators of the FAIR Epigraphy Project. We sent the survey to 16
partner projects. We received 13 responses to the survey, with a
response rate of 75\% with some participants responding on behalf of two
projects combined into one response (and thus skewing the response
rate). 92\% of partner projects gave consent to publish their anonymised
responses as part of this report. Those who participated in the survey
but did not give their consent are excluded from the report.

The partner projects represent relatively established projects with the
average duration of a project being 6 years. The shortest participating
project reported their duration as 3 years and the longest 207 as years.

\hypertarget{language-coverage}{%
\subsection{Language coverage}\label{language-coverage}}

\textbf{Question:} \emph{What is the predominant language of epigraphic
data in your project (for mixed collections or collections where other
languages are predominant provide details in Other)}

\begin{longtable}[]{@{}
  >{\raggedright\arraybackslash}p{(\columnwidth - 4\tabcolsep) * \real{0.8750}}
  >{\raggedleft\arraybackslash}p{(\columnwidth - 4\tabcolsep) * \real{0.0417}}
  >{\raggedleft\arraybackslash}p{(\columnwidth - 4\tabcolsep) * \real{0.0833}}@{}}
\toprule
\begin{minipage}[b]{\linewidth}\raggedright
language
\end{minipage} & \begin{minipage}[b]{\linewidth}\raggedleft
n
\end{minipage} & \begin{minipage}[b]{\linewidth}\raggedleft
ratio
\end{minipage} \\
\midrule
\endhead
Latin & 8 & 28 \\
Greek & 7 & 24 \\
Hebrew & 2 & 7 \\
Other & 2 & 7 \\
Ancient Celtic & 1 & 3 \\
Elymian & 1 & 3 \\
Etruscan & 1 & 3 \\
Gaulish & 1 & 3 \\
Oscan & 1 & 3 \\
other epichoric languages from the west provinces (ex. Africa) & 1 &
3 \\
Phoenician-Punic & 1 & 3 \\
Punic & 1 & 3 \\
Raetic & 1 & 3 \\
Sikel & 1 & 3 \\
\bottomrule
\end{longtable}

\textbf{Commentary:} The language coverage of the participating projects
consists predominantly of Latin and Greek either on its own or in
combination (representing 52\% of the answers). The languages listed as
\texttt{Other} consisted of \emph{Other, Ancient Celtic, Elymian,
Etruscan, Gaulish, Oscan, other epichoric languages from the west
provinces (ex. Africa)}. 7 participating projects record inscriptions in
one language only, while 5 contain inscriptions in two and more
languages (7 being the maximum number of listed languages.) The
\texttt{Other} (i.e.~other than Greek and Latin) category encompassed a
substantial part of the surveyed projects, documenting the need to
expand beyond the traditional Latin and Greek focus of the classical
epigraphic discipline. It is, however, worth noting the majority of
participating projects the records languages from the wider
Mediterranean/European linguistic space.

\textbf{Combinations of languages as retrieved from the survey:}

\begin{verbatim}
##  [1] "Greek; Latin"                                                                
##  [2] "Latin"                                                                       
##  [3] "Gaulish"                                                                     
##  [4] "Latin; Greek; Punic; Etruscan; Hebrew; Raetic; Other"                        
##  [5] "Latin; Greek; other epichoric languages from the west provinces (ex. Africa)"
##  [6] "Greek"                                                                       
##  [7] "Ancient Celtic"                                                              
##  [8] "Latin"                                                                       
##  [9] "Latin; Greek; Other"                                                         
## [10] "Greek; Latin; Phoenician-Punic; Oscan; Sikel; Elymian; Hebrew"               
## [11] "Latin"                                                                       
## [12] "Greek"
\end{verbatim}

\begin{center}\rule{0.5\linewidth}{0.5pt}\end{center}

\hypertarget{it-infrastructure}{%
\subsection{IT infrastructure}\label{it-infrastructure}}

\textbf{Question}: \emph{Does the project have a website?}

\begin{longtable}[]{@{}lr@{}}
\toprule
Website & n \\
\midrule
\endhead
Yes & 12 \\
\bottomrule
\end{longtable}

\textbf{Commentary}: All of the participating projects currently
maintain an online presence (as of February 2022).

\begin{center}\rule{0.5\linewidth}{0.5pt}\end{center}

\textbf{Question}: \emph{Does your project have an IT specialist(s)?}

To see more data, click on the arrow in the top right hand corner of the
table below.

\begin{longtable}[]{@{}
  >{\raggedright\arraybackslash}p{(\columnwidth - 4\tabcolsep) * \real{0.9492}}
  >{\raggedleft\arraybackslash}p{(\columnwidth - 4\tabcolsep) * \real{0.0169}}
  >{\raggedleft\arraybackslash}p{(\columnwidth - 4\tabcolsep) * \real{0.0339}}@{}}
\toprule
\begin{minipage}[b]{\linewidth}\raggedright
IT\_spec
\end{minipage} & \begin{minipage}[b]{\linewidth}\raggedleft
n
\end{minipage} & \begin{minipage}[b]{\linewidth}\raggedleft
ratio
\end{minipage} \\
\midrule
\endhead
Yes, equivalent of part-time (\textless1.0 FTE) position & 7 & 58 \\
No & 2 & 17 \\
We paid some specialists, but currently we have no budget for them (and
this is a problem, even for the sustainability and ordinary maintainance
of our digital assets) & 1 & 8 \\
Yes, equivalent of full-time (1.0 FTE) position & 1 & 8 \\
Yes, equivalent of more than full-time (\textgreater1.0 FTE) position &
1 & 8 \\
\bottomrule
\end{longtable}

\textbf{Commentary}: Majority of these established digital projects have
an IT specialist, yet only 2 projects have an equivalent of 1.0 FTE or
more at their disposal. 58 \% of projects have access to part-time IT
support for their projects, which in some instances may be only a few
hours per week per project. 3 projects reported no availability of IT
support, even on a part-time basis (representing 25\% of participating
partner projects).

\begin{center}\rule{0.5\linewidth}{0.5pt}\end{center}

\textbf{Question}: \emph{Does your project store epigraphic data in the
following formats\ldots?}

\begin{longtable}[]{@{}lrr@{}}
\toprule
format & n & no\_format \\
\midrule
\endhead
Epidoc XML & 4 & 1 \\
Epidoc XML, CSV, SQL or similar & 1 & 3 \\
Epidoc XML, in print & 1 & 2 \\
Epidoc XML, JSON, CSV & 1 & 3 \\
Epidoc XML, JSON, RDF & 1 & 3 \\
Epidoc XML, SQL or similar & 1 & 2 \\
RDF, SQL or similar & 1 & 2 \\
SQL or similar & 1 & 1 \\
We aim to switch to Epidoc XML storable data & 1 & 1 \\
\bottomrule
\end{longtable}

\textbf{Commentary}: The majority of projects use \texttt{Epidoc\ XML}
as their main output data format (83\% of participating projects),
either in combination with other formats or as a sole data format. Other
data formats are represented less frequently: JSON (17\%), RDF (17\%),
SQL(33\%) and CSV (17\%). 25\% of projects use only one type of data
format, while 50\% use two or more data format types.

\hypertarget{data-sharing}{%
\subsection{Data sharing}\label{data-sharing}}

\textbf{Question}: Do you share your data outside of your project?

\begin{longtable}[]{@{}
  >{\raggedright\arraybackslash}p{(\columnwidth - 4\tabcolsep) * \real{0.8929}}
  >{\raggedleft\arraybackslash}p{(\columnwidth - 4\tabcolsep) * \real{0.0357}}
  >{\raggedleft\arraybackslash}p{(\columnwidth - 4\tabcolsep) * \real{0.0714}}@{}}
\toprule
\begin{minipage}[b]{\linewidth}\raggedright
share
\end{minipage} & \begin{minipage}[b]{\linewidth}\raggedleft
n
\end{minipage} & \begin{minipage}[b]{\linewidth}\raggedleft
ratio
\end{minipage} \\
\midrule
\endhead
Yes, under a Creative Commons license & 7 & 58 \\
Not currently, but we are thinking about it & 1 & 8 \\
Yes, on demand & 1 & 8 \\
Yes, without any license & 1 & 8 \\
Yes, without any license, We aim to switch to CC-BY 4.0 & 1 & 8 \\
Yes, without any licenses, The question of licences is under
consideration & 1 & 8 \\
\bottomrule
\end{longtable}

\textbf{Commentary}: All partner projects reported their willingness to
share the data, even if they are not currently doing it, or if they
provide the data only on demand. 58\% of partner projects share the data
under a Creative Commons license (\url{https://creativecommons.org/}),
which is the preferred mode according to the FAIR data principles. Large
part of partners that are currently not using Creative Commons licenses
are considering their use in the future.

\begin{center}\rule{0.5\linewidth}{0.5pt}\end{center}

\textbf{Question}: \emph{How do share your data with users outside your
project?}

To see more data, click on the arrow in the top right hand corner of the
table below.

\begin{longtable}[]{@{}
  >{\raggedright\arraybackslash}p{(\columnwidth - 4\tabcolsep) * \real{0.9574}}
  >{\raggedleft\arraybackslash}p{(\columnwidth - 4\tabcolsep) * \real{0.0080}}
  >{\raggedleft\arraybackslash}p{(\columnwidth - 4\tabcolsep) * \real{0.0346}}@{}}
\toprule
\begin{minipage}[b]{\linewidth}\raggedright
share\_all
\end{minipage} & \begin{minipage}[b]{\linewidth}\raggedleft
n
\end{minipage} & \begin{minipage}[b]{\linewidth}\raggedleft
share\_method
\end{minipage} \\
\midrule
\endhead
Individual Epidoc XMLs or Epidoc XML dumps on the website & 1 & 1 \\
Individual Epidoc XMLs or Epidoc XML dumps on the website; Via search
output on our website & 1 & 2 \\
Other publicly accessible repository (specify in Other); Universitat de
Barcelona & 1 & 2 \\
Public repository on GitHub & 1 & 1 \\
Public repository on GitHub; Individual Epidoc XMLs or Epidoc XML dumps
on the website; We have an API but not documented or made public & 1 &
3 \\
Public repository on GitHub; Zenodo; Individual JSONs or JSON dump on
the website; Individual CSVs or CSV dumps on the website; Individual
Epidoc XMLs or Epidoc XML dumps on the website; Via search output on our
website; Not all of the above currently functioning but will be by end
of project. The main public visualisation of the data will be in the
webGIS. & 1 & 7 \\
Public repository on GitHub; Zenodo; Other publicly accessible
repository (specify in Other); Individual CSVs or CSV dumps on the
website; Individual Epidoc XMLs or Epidoc XML dumps on the website; Via
search output on our website; also on University repository; and geo
data in rdf static dump to Pelagios & 1 & 8 \\
Via search output on our website; Other output forms are presently under
consideration & 1 & 2 \\
Via search output on our website; We aim to switch to more technical
dumps & 1 & 2 \\
Via search output on our website; We sent an email with requested data &
1 & 2 \\
Zenodo; Individual CSVs or CSV dumps on the website & 1 & 2 \\
Zenodo; Individual JSONs or JSON dump on the website; Individual CSVs or
CSV dumps on the website; Individual Epidoc XMLs or Epidoc XML dumps on
the website & 1 & 4 \\
\bottomrule
\end{longtable}

\textbf{Commentary}: All partner projects provide at least one way of
sharing the data (whether it may be currently accessible to the public
or not, or it is intended to be accessible in the future). The average
(median) number of sharing methods per project is 2.

\begin{figure}

{\centering \includegraphics{01_FAIR_epi_report_files/figure-latex/unnamed-chunk-11-1} 

}

\caption{Figure showing the popularity of individual sharing methods and formats across partner projects.}\label{fig:unnamed-chunk-11}
\end{figure}

\textbf{Commentary}: Epidoc XML is by far the most popular format for
data sharing (implemented by 6 projects), however other Open Science
services are starting to make their way into established digital
epigraphy projects, such as sharing via a public repository, implemented
by 4 (GitHub) and 4 (Zenodo) projects respectively, as well as providing
raw data in the CSV (comma-separated value) format (4 projects), or as
JSON (JavaScript Object Notation) files (4 projects). Only a relative
minority of participating partner projects shares the data on an
on-demand basis or have a non-public API access point to their data.

\hypertarget{institutional-policies}{%
\subsection{Institutional policies}\label{institutional-policies}}

\textbf{Question:} \emph{Does your institution or funding body require
your project to comply with any data policies (e.g., FAIR principles,
data storage, data sharing, Open Science)?}

\begin{longtable}[]{@{}
  >{\raggedright\arraybackslash}p{(\columnwidth - 4\tabcolsep) * \real{0.9091}}
  >{\raggedleft\arraybackslash}p{(\columnwidth - 4\tabcolsep) * \real{0.0303}}
  >{\raggedleft\arraybackslash}p{(\columnwidth - 4\tabcolsep) * \real{0.0606}}@{}}
\toprule
\begin{minipage}[b]{\linewidth}\raggedright
policies
\end{minipage} & \begin{minipage}[b]{\linewidth}\raggedleft
n
\end{minipage} & \begin{minipage}[b]{\linewidth}\raggedleft
ratio
\end{minipage} \\
\midrule
\endhead
Yes & 7 & 58 \\
No & 3 & 25 \\
Not yet & 1 & 8 \\
The ERC open data policies don't apply to this project, but we are
following them anyway. & 1 & 8 \\
\bottomrule
\end{longtable}

\textbf{Commentary}: The majority of projects (represented by 58\%) are
required to comply with data-related policy introduced either by their
institution or a funding body. A smaller part of partner projects (33\%)
is not required to follow any data policy, but some follow it on a
voluntary basis.

\begin{center}\rule{0.5\linewidth}{0.5pt}\end{center}

\textbf{Question}: \emph{If you have answered YES in the previous
question, please specify what are the policies, or provide a link.}

\begin{verbatim}
## [1] "The funding body (the ERC) expects that research results are available in open
access"
## [2] "Open access publication."
## [3] "Actualització de la Política d’Accés Obert a la Universitat de Barcelona
(http://hdl.handle.net/2445/142065)"
## [4] "Data Sharing, Open Science"
## [5] "usual ERC requirements"
## [6] "Currently subject to ERC open data policies
(https://ec.europa.eu/research/participants/docs/h2020-funding-guide/cross-cutting-issues/open-access-data-management/open-access_en.htm);
previously Oxford University open data policies (which mirror UKRI policies:
https://www.ukri.org/publications/ukri-open-access-policy/)"
## [7] "https://forschungsdatenmanagement.bbaw.de/de"
\end{verbatim}

\textbf{Commentary}: Several of the partner projects follow the ERC data
and open access policies. More information on ERC Open Research Data and
Data Management Plans can be found at
\url{https://erc.europa.eu/sites/default/files/document/file/ERC_info_document-Open_Research_Data_and_Data_Management_Plans.pdf}
or at ERC Open Science policies page
\url{https://erc.europa.eu/managing-your-project/open-science}.

\hypertarget{open-science-practice}{%
\subsection{Open Science practice}\label{open-science-practice}}

\textbf{Question}: \emph{Standardized terminologies: The project uses
the following systems:}

To see more data, click on the arrow in the top right hand corner of the
table below.

\begin{longtable}[]{@{}
  >{\raggedright\arraybackslash}p{(\columnwidth - 4\tabcolsep) * \real{0.9615}}
  >{\raggedleft\arraybackslash}p{(\columnwidth - 4\tabcolsep) * \real{0.0128}}
  >{\raggedleft\arraybackslash}p{(\columnwidth - 4\tabcolsep) * \real{0.0256}}@{}}
\toprule
\begin{minipage}[b]{\linewidth}\raggedright
standard\_terminologies
\end{minipage} & \begin{minipage}[b]{\linewidth}\raggedleft
n
\end{minipage} & \begin{minipage}[b]{\linewidth}\raggedleft
ratio
\end{minipage} \\
\midrule
\endhead
Internal authority lists & 7 & 58 \\
Own version of EAGLE vocabularies (edited for our project) & 7 & 58 \\
EAGLE vocabularies as provided at
\url{https://www.eagle-network.eu/resources/vocabularies/} & 5 & 42 \\
``Pre-defined lists are a problem in itself as it is difficult to
categorize in advance everything what may be found in real life. Besides
there often are open questions. Flexibility and openness for real
evidence is needed.'' & 1 & 8 \\
\url{http://kerameikos.org/} for vase forms & 1 & 8 \\
some use of \url{https://www.getty.edu/research/tools/vocabularies/aat/}
where EAGLE is deficient & 1 & 8 \\
\bottomrule
\end{longtable}

\textbf{Commentary}: The lists of vocabularies for the epigraphic
discipline created by the EAGLE project
(\url{https://www.eagle-network.eu/resources/vocabularies/}) are used by
most of the projects: either in the original form (50\% of participating
projects) or in the form modified for the needs of the project (66\% of
participating projects). The need for modifications suggests that the
EAGLE vocabularies do not in fact form a community-wide standard and
need to be improved before becoming one. The process has been already
started by the Epigraphy.info Vocabularies working group of which
Hermankova, Horster, and Prag are all members. For more details see
\url{https://epigraphy.info/vocabularies_wg/}. If you would like to join
the working group, please get in touch with the authors.

\begin{center}\rule{0.5\linewidth}{0.5pt}\end{center}

\textbf{Question}: \emph{Standardized terminologies: data on combination
of vocabularies systems}

\begin{longtable}[]{@{}rrr@{}}
\toprule
standard\_method\_no & n & ratio \\
\midrule
\endhead
1 & 6 & 50 \\
2 & 3 & 25 \\
3 & 2 & 17 \\
4 & 1 & 8 \\
\bottomrule
\end{longtable}

\textbf{Commentary}: The majority of projects (50\%) uses only one
method to record their standard terminologies, while 50\% of projects
use a combination of two or three methods. Internal authority lists are
used in combination with the EAGLE vocabularies both in their original
and modified form. Sharing or publication of internal authority lists
would therefore be highly beneficial for improving the existing EAGLE
vocabularies.

\begin{center}\rule{0.5\linewidth}{0.5pt}\end{center}

\textbf{Question}: \emph{Linked Open Datasets: The project uses the
following systems:}

\begin{longtable}[]{@{}
  >{\raggedright\arraybackslash}p{(\columnwidth - 6\tabcolsep) * \real{0.7895}}
  >{\raggedleft\arraybackslash}p{(\columnwidth - 6\tabcolsep) * \real{0.0263}}
  >{\raggedleft\arraybackslash}p{(\columnwidth - 6\tabcolsep) * \real{0.0526}}
  >{\raggedleft\arraybackslash}p{(\columnwidth - 6\tabcolsep) * \real{0.1316}}@{}}
\toprule
\begin{minipage}[b]{\linewidth}\raggedright
linked\_data
\end{minipage} & \begin{minipage}[b]{\linewidth}\raggedleft
n
\end{minipage} & \begin{minipage}[b]{\linewidth}\raggedleft
ratio
\end{minipage} & \begin{minipage}[b]{\linewidth}\raggedleft
ratio\_all\_proj
\end{minipage} \\
\midrule
\endhead
Trismegistos & 10 & 24 & 83 \\
Pleiades & 9 & 22 & 75 \\
EAGLE vocabularies & 7 & 17 & 58 \\
PIR & 3 & 7 & 25 \\
EDH People & 2 & 5 & 17 \\
LGPN & 2 & 5 & 17 \\
Adriatlas & 1 & 2 & 8 \\
Cartapulia & 1 & 2 & 8 \\
\url{http://kerameikos.org/} & 1 & 2 & 8 \\
More under consideration. Trismegistos is not free! & 1 & 2 & 8 \\
OxREP mines database & 1 & 2 & 8 \\
Period.O & 1 & 2 & 8 \\
We provide TM references in our bibliography but inconsistently and
without cross linking & 1 & 2 & 8 \\
We were working on a cooperation with TM when our funding finished & 1 &
2 & 8 \\
\bottomrule
\end{longtable}

\textbf{Commentary}: From the listed Linked Open Datasets (LOD),
Pleaides and Trismegistos are by far the most popular, being used in
75\% and 83\% of all participating projects. The EAGLE vocabularies are
used in 58\% of all participating projects. Prosopographic data,
represented by EDH People, PIR, and LGPN are used by 59\% of all
participating projects. The survey responses suggest there is a great
space for improvement and potentially great benefit in creating and
further improving prosopographic LOD.

\hypertarget{non-partnered-epigraphy-projects}{%
\section{Non-partnered epigraphy
projects}\label{non-partnered-epigraphy-projects}}

This section summarises the results of the online survey
\href{https://github.com/FAIR-epigraphy/scoping_survey_report/data/02_Survey_projects_questions.pdf}{\emph{Digital
epigraphy in 2022: scoping survey}} for all digital epigraphy projects)
aimed at digital projects currently listed under Digital Epigraphy
Projects on the Digital Classicist Wiki page
(\url{https://wiki.digitalclassicist.org/Category:Projects}) that were
possible to trace in February 2022. The survey was sent to 83 projects
and the link circulated until mid-April 2022. We have received 27
responses to the survey, a response rate of 31\%. Some participants
contributed on behalf of multiple projects in one response, which we
were unable to disentangle and thus the response rate is slightly
skewed. 96\% of non-partnered projects gave consent to publish their
anonymised responses as part of the current report. The remaining
responses are excluded from the report but will be used to inform the
FAIR Epigraphy planning and decision making.

The respondents represent a wide range of projects from well established
projects to short-term mostly PhD projects, with the average duration of
a project being 5.5 years. The shortest participating project reported
their duration as 1 year and the longest 117 years (that was clearly not
digital for the whole of that time).

\textbf{Question}: \emph{Is the project still active?}

\begin{longtable}[]{@{}lrr@{}}
\toprule
status & n & ratio \\
\midrule
\endhead
Yes & 19 & 73 \\
Currently not, but we are considering a re-start & 4 & 15 \\
No, the project is closed & 3 & 12 \\
\bottomrule
\end{longtable}

\textbf{Commentary}: 73\% of responding projects are still active, while
12\% of projects are permanently closed and do not consider restarting
in the future. 15\% of projects are currently not active, but might be
reactivated in the future.

\hypertarget{language-coverage-1}{%
\subsection{Language coverage}\label{language-coverage-1}}

\textbf{Question:} \emph{What is the predominant language of epigraphic
data in your project (for mixed collections or collections where other
languages are predominant provide details in Other)}

\begin{longtable}[]{@{}lrr@{}}
\toprule
language & n & ratio \\
\midrule
\endhead
Greek & 16 & 38 \\
Latin & 11 & 26 \\
Phoenician & 2 & 5 \\
Akkadian & 1 & 2 \\
Ancient Languages of the Mediterranean area & 1 & 2 \\
Arabic & 1 & 2 \\
Aramaic & 1 & 2 \\
Hattian u.a. & 1 & 2 \\
Hittite & 1 & 2 \\
Hurrian & 1 & 2 \\
Luwian & 1 & 2 \\
Neopunic & 1 & 2 \\
Other & 1 & 2 \\
Palaeo-European & 1 & 2 \\
Palaeo-Hispanic & 1 & 2 \\
Punic & 1 & 2 \\
\bottomrule
\end{longtable}

\textbf{Commentary:} The language coverage of the participating projects
consisted predominantly of Latin and Greek projects representing 64\% of
projects combined. Greek being the most frequent language, either as a
sole/predominant language (11 projects) or in combination with other
languages (5 projects). Latin being a sole/predominant language in 6
projects or in combination with other languages (5 projects). 18
participating projects record inscriptions in one language only, while 8
contain inscriptions in two and more languages (5 being the maximum
number of listed languages.).

The languages listed as \texttt{Other} consisted of languages such as
Phoenician, Akkadian, Ancient Languages of the Mediterranean area,
Arabic, Aramaic, Hattian u.a., Hittite, Hurrian, Luwian, Neopunic,
Other, Palaeo-European, Palaeo-Hispanic, Punic. All languages come from
the wider Mediterranean/European linguistic space.

\textbf{Combinations of languages as retrieved from the survey:}

\begin{verbatim}
##  [1] "Latin"                                           
##  [2] "Greek"                                           
##  [3] "Greek; Latin; Other"                             
##  [4] "Latin; Greek"                                    
##  [5] "Hittite; Akkadian; Hurrian; Luwian; Hattian u.a."
##  [6] "Greek; Latin; Aramaic; Phoenician; Arabic"       
##  [7] "Greek; Latin"                                    
##  [8] "Phoenician; Punic; Neopunic"                     
##  [9] "Palaeo-European; Palaeo-Hispanic"                
## [10] "Ancient Languages of the Mediterranean area"
\end{verbatim}

\hypertarget{it-infrastructure-1}{%
\subsection{IT infrastructure}\label{it-infrastructure-1}}

\textbf{Question}: \emph{Does the project have a website?}

\begin{longtable}[]{@{}lr@{}}
\toprule
Website & n \\
\midrule
\endhead
Yes & 25 \\
No & 1 \\
\bottomrule
\end{longtable}

\textbf{Commentary}: The vast majority of the participating projects
maintains an online presence (as of February 2022).

\begin{center}\rule{0.5\linewidth}{0.5pt}\end{center}

\textbf{Question}: \emph{Does your project have an IT specialist(s)?}

To see more data, click on the arrow in the top right hand corner of the
table below.

\begin{longtable}[]{@{}
  >{\raggedright\arraybackslash}p{(\columnwidth - 4\tabcolsep) * \real{0.9455}}
  >{\raggedleft\arraybackslash}p{(\columnwidth - 4\tabcolsep) * \real{0.0182}}
  >{\raggedleft\arraybackslash}p{(\columnwidth - 4\tabcolsep) * \real{0.0364}}@{}}
\toprule
\begin{minipage}[b]{\linewidth}\raggedright
IT\_spec
\end{minipage} & \begin{minipage}[b]{\linewidth}\raggedleft
n
\end{minipage} & \begin{minipage}[b]{\linewidth}\raggedleft
ratio
\end{minipage} \\
\midrule
\endhead
N/A & 7 & 27 \\
No & 6 & 23 \\
Yes, equivalent of part-time (\textless1.0 FTE) position & 6 & 23 \\
Yes, equivalent of full-time (1.0 FTE) position & 2 & 8 \\
depending on development steps; expertise and experience transfer also
among project staff. & 1 & 4 \\
We are in cooperation with an IT specialist (equivalent of full-time
(1.0 FTE) position) of another project who takes care of a couple of
databases. & 1 & 4 \\
We do not have an IT specialist permanently assigned to the project, but
the project has institutional support, including whatever IT support is
necessary. & 1 & 4 \\
We had & 1 & 4 \\
We have the support of two IT specialists for maintenance and small
updates, but for every major development we need to find new funding & 1
& 4 \\
\bottomrule
\end{longtable}

\textbf{Commentary}: Only 8\% of projects have an equivalent of 1.0 FTE
or more at their disposal. 35\% of digital projects have an IT
specialist available for at least several hours per week or share them
with other digital projects within their institution. Several projects
report difficulty with finding financial resources to support further
development and long-term sustainability of the project or even
day-to-day support. 27\% of the participating projects report that they
currently do not have any access to IT support. An additional 27\% of
projects did not indicate whether they have access to IT support because
they are no longer active. In order to understand the precise
significance of this data, it would be necessary in future surveys to
clarify the current funding status of individual projects.

\begin{center}\rule{0.5\linewidth}{0.5pt}\end{center}

\textbf{Question}: \emph{Does your project store epigraphic data in the
following formats\ldots?}

To see more data, click on the arrow in the top right hand corner of the
table below.

\begin{longtable}[]{@{}
  >{\raggedright\arraybackslash}p{(\columnwidth - 4\tabcolsep) * \real{0.9261}}
  >{\raggedleft\arraybackslash}p{(\columnwidth - 4\tabcolsep) * \real{0.0170}}
  >{\raggedleft\arraybackslash}p{(\columnwidth - 4\tabcolsep) * \real{0.0568}}@{}}
\toprule
\begin{minipage}[b]{\linewidth}\raggedright
format
\end{minipage} & \begin{minipage}[b]{\linewidth}\raggedleft
n
\end{minipage} & \begin{minipage}[b]{\linewidth}\raggedleft
no\_format
\end{minipage} \\
\midrule
\endhead
Epidoc XML & 6 & 1 \\
SQL or similar & 4 & 1 \\
CSV, SQL or similar & 1 & 2 \\
Epidoc XML, CSV & 1 & 2 \\
Epidoc XML, JSON, RDF, CSV & 1 & 4 \\
Epidoc XML, SQL or similar & 1 & 2 \\
JSON, SQL or similar, the xml version of the data is available through
the EAGLE project & 1 & 3 \\
None - we use analog systems (printed), 3d viewers & 1 & 2 \\
RDF & 1 & 1 \\
SQL or similar, We are working on providing also an Epidoc XML version
of at least the annotated texts
(\url{https://epidoc.stoa.org/gl/latest/app-epi-mycenaean.html})) & 1 &
2 \\
XML adapted from Epidoc XML & 1 & 1 \\
\bottomrule
\end{longtable}

\textbf{Commentary}: The majority of projects use \texttt{Epidoc\ XML}
as their main output data format (42\% of participating projects),
either in combination with other formats or as a sole data format. SQL
and similar database formats are relatively common in 31\% of projects.
Other data formats are represented less frequently by a small number of
projects and mostly as complementary data formats to more popular
formats such as Epidoc XML or SQL: JSON (8\%), RDF (8\%), and CSV
(12\%). 4\% of projects indicated the use of combination of analogue
data and 3D data format. 4\% of projects indicated using their own
version of Epidoc XML, adapted to their specific needs.

15\% of projects use only one type of data format, while 27\% use two or
more data format types (such as Epidoc XML, SQL or similar, CSV, JSON,
RDF, the xml version of the data is available through the EAGLE project,
None - we use analog systems (printed), 3d viewers, We are working on
providing also an Epidoc XML version of at least the annotated texts
(\url{https://epidoc.stoa.org/gl/latest/app-epi-mycenaean.html})), XML
adapted from Epidoc XML).

The frequent use of SQL format signalize a relatively low compliance
with the FAIR data principles, with individual databases being recorded
in non-standard format with a specific purpose in mind, which are not
immediately interoperable with e.g.~Epidoc XML based projects. 4
projects use SQL as their sole data storage format, and depending on the
structure of a given database, there might be a higher risk of not being
easily \emph{Findable}, \emph{Accessible}, and \emph{Interoperable} with
the rest of the existing epigraphic datasets.

The data format of the projects that are no longer active is recorded in
the following \emph{Data sharing} section, under \emph{Closed Projects}.

\hypertarget{data-sharing-1}{%
\subsection{Data sharing}\label{data-sharing-1}}

\hypertarget{active-projects}{%
\subsubsection{Active projects}\label{active-projects}}

This section summarized only the `active' projects. For
`closed/non-active' projects, see the section below.

\textbf{Question}: \emph{Do you share your data outside of your
project?}

To see more data, click on the arrow in the top right hand corner of the
table below.

\begin{longtable}[]{@{}
  >{\raggedright\arraybackslash}p{(\columnwidth - 4\tabcolsep) * \real{0.9217}}
  >{\raggedleft\arraybackslash}p{(\columnwidth - 4\tabcolsep) * \real{0.0261}}
  >{\raggedleft\arraybackslash}p{(\columnwidth - 4\tabcolsep) * \real{0.0522}}@{}}
\toprule
\begin{minipage}[b]{\linewidth}\raggedright
share
\end{minipage} & \begin{minipage}[b]{\linewidth}\raggedleft
n
\end{minipage} & \begin{minipage}[b]{\linewidth}\raggedleft
ratio
\end{minipage} \\
\midrule
\endhead
Yes, under a Creative Commons license & 8 & 42 \\
Not currently, but we are thinking about it & 3 & 16 \\
so far without explicit license & 1 & 5 \\
Under demand & 1 & 5 \\
we periodically share our data with the Europeana platform & 1 & 5 \\
Yes, publishing contributions with link to the Catalogue of the projects
& 1 & 5 \\
Yes, under a Creative Commons license, and also French Etalab Licence
Ouverte / Open Licence & 1 & 5 \\
Yes, under a Creative Commons license, by login through guest password &
1 & 5 \\
Yes, under a Creative Commons license, We are linked with other
databases (Clauss \& Slabby, for instance) & 1 & 5 \\
Yes, without any license & 1 & 5 \\
\bottomrule
\end{longtable}

\textbf{Commentary}: As of February 2022, 19 projects participated in
the survey as active projects. The majority of active projects are
willing to share their data, representing 82\% of participating
projects. 57\% of active projects share the data under a Creative
Commons license, which is the preferred mode according to the FAIR data
principles. 10\% of active projects share the data without any specific
license, while 5\% provide the data only on demand.

\begin{center}\rule{0.5\linewidth}{0.5pt}\end{center}

\textbf{Question}: \emph{How do share your data with users outside your
project?}

To see more data, click on the arrow in the top right hand corner of the
table below.

\begin{longtable}[]{@{}
  >{\raggedright\arraybackslash}p{(\columnwidth - 4\tabcolsep) * \real{0.9454}}
  >{\raggedleft\arraybackslash}p{(\columnwidth - 4\tabcolsep) * \real{0.0102}}
  >{\raggedleft\arraybackslash}p{(\columnwidth - 4\tabcolsep) * \real{0.0444}}@{}}
\toprule
\begin{minipage}[b]{\linewidth}\raggedright
share\_all
\end{minipage} & \begin{minipage}[b]{\linewidth}\raggedleft
n
\end{minipage} & \begin{minipage}[b]{\linewidth}\raggedleft
share\_method
\end{minipage} \\
\midrule
\endhead
Individual Epidoc XMLs or Epidoc XML dumps on the website & 2 & 1 \\
Via search output on our website & 2 & 1 \\
depending of the request & 1 & 1 \\
Individual CSVs or CSV dumps on the website & 1 & 1 \\
Individual Epidoc XMLs or Epidoc XML dumps on the forthcoming website
and GitHub & 1 & 1 \\
Individual Epidoc XMLs or Epidoc XML dumps on the website, We sent an
email with requested data & 1 & 2 \\
on request & 1 & 1 \\
Other publicly accessible repository (specify in Other),
\url{http://repository.edition-topoi.org/collection/ICG} & 1 & 2 \\
Other publicly accessible repository (specify in Other), Individual
JSONs or JSON dump on the website, Individual Epidoc XMLs or Epidoc XML
dumps on the website, Public API on our website, French Huma-Num
platform and services, particularly Nakala services for our photographs
& 1 & 6 \\
Other publicly accessible repository (specify in Other), We have a
DSpace instance for sharing project data. At present we are behind on
putting material on the externally accessible database because of
complications in moving to an EpiDoc-based metadata system. & 1 & 2 \\
Public repository on GitHub, Individual Epidoc XMLs or Epidoc XML dumps
on the website & 1 & 2 \\
Sketchfab website & 1 & 1 \\
Via search output on our website, We sent an email with requested data,
Research Data Repository (RDR) of the Cluster of Excellence
``Understanding Written Artefacts''; Individual Epidoc XMLs or Epidoc
XML dumps on the website is planned for the future. & 1 & 4 \\
We don't currently share data outside our project & 1 & 1 \\
We sent an email with requested data, We ar planning to have an API;
Search results can be downloaded as CSV files. & 1 & 3 \\
Zenodo, Other publicly accessible repository (specify in Other),
Individual Epidoc XMLs or Epidoc XML dumps on the website & 1 & 3 \\
Zenodo, the xml version of the data is available through the EAGLE
project & 1 & 2 \\
\bottomrule
\end{longtable}

\textbf{Commentary}: As of February 2022, all active projects provide at
least one way of sharing the data (whether it is currently accessible to
the public or not, or it is intended to be accessible in the future).
The average (median) number of sharing methods per project is 2, while
the maximum number is 6 (e.g.,Other publicly accessible repository
(specify in Other), Individual JSONs or JSON dump on the website,
Individual Epidoc XMLs or Epidoc XML dumps on the website, Public API on
our website, French Huma-Num platform and services, particularly Nakala
services for our photographs).

There is no discipline-wide standard for data repository as all projects
use either their institutional or national resources that may or may not
be ideal for epigraphic data. From those who share the data, the Epidoc
XML format is the most popular format for data sharing, as well as
search output on the project's website. Open Science practices do not
seem to be a popular choice in digital epigraphy, such as sharing via
public repository (e.g., GitHub or Zenodo), as well as providing raw
data in the CSV (comma-separated value) format, or as JSON (JavaScript
Object Notation) files. Computer-automated access to data is rare and
manual human interaction, such as manual selection and/or manual
download of files prevails, potentially hindering any quantitative and
reproducible studies, or linking of datasets via automated processes.
For example, an API access point is currently available only for a very
limited number of projects.

\begin{center}\rule{0.5\linewidth}{0.5pt}\end{center}

\hypertarget{closed-projects}{%
\subsubsection{Closed projects}\label{closed-projects}}

This section summarized only the `closed/non-active' projects. For
`active' projects, see the section above.

\textbf{Question}: \emph{Is the data created by your project
accessible?}

\begin{longtable}[]{@{}
  >{\raggedright\arraybackslash}p{(\columnwidth - 4\tabcolsep) * \real{0.8714}}
  >{\raggedleft\arraybackslash}p{(\columnwidth - 4\tabcolsep) * \real{0.0429}}
  >{\raggedleft\arraybackslash}p{(\columnwidth - 4\tabcolsep) * \real{0.0857}}@{}}
\toprule
\begin{minipage}[b]{\linewidth}\raggedright
share
\end{minipage} & \begin{minipage}[b]{\linewidth}\raggedleft
n
\end{minipage} & \begin{minipage}[b]{\linewidth}\raggedleft
ratio
\end{minipage} \\
\midrule
\endhead
Yes, under a Creative Commons license & 5 & 71 \\
Not currently, but we are thinking about making it available & 1 & 14 \\
Yes, without any license & 1 & 14 \\
\bottomrule
\end{longtable}

\textbf{Commentary}: As of February 2022, 7 of the participating
projects are closed. 71\% of them provides access to their data under a
Creative Commons license even though the project is no longer active,
14\% of closed projects provide access without any license and 14\% do
not currently provide access to the data they have created during the
duration of their project, but they are considering to make the data
available.

\begin{center}\rule{0.5\linewidth}{0.5pt}\end{center}

\textbf{Question}: \emph{Is the data created by your project
accessible?}

\begin{longtable}[]{@{}
  >{\raggedright\arraybackslash}p{(\columnwidth - 4\tabcolsep) * \real{0.9011}}
  >{\raggedleft\arraybackslash}p{(\columnwidth - 4\tabcolsep) * \real{0.0330}}
  >{\raggedleft\arraybackslash}p{(\columnwidth - 4\tabcolsep) * \real{0.0659}}@{}}
\toprule
\begin{minipage}[b]{\linewidth}\raggedright
service
\end{minipage} & \begin{minipage}[b]{\linewidth}\raggedleft
n
\end{minipage} & \begin{minipage}[b]{\linewidth}\raggedleft
ratio
\end{minipage} \\
\midrule
\endhead
Individual Epidoc XMLs or Epidoc XML dumps on the website & 4 & 57 \\
Public repository on GitHub & 3 & 43 \\
Other publicly accessible repository (specify in Other) & 2 & 29 \\
\url{https://dspace-clarin-it.ilc.cnr.it/repository/xmlui/handle/20.500.11752/OPEN-548}
& 1 & 14 \\
\url{https://open.library.ubc.ca/collections/squeezes} & 1 & 14 \\
ILC4CLARIN Repository & 1 & 14 \\
Via search output on our website & 1 & 14 \\
We don't currently share data outside our project & 1 & 14 \\
\bottomrule
\end{longtable}

\textbf{Commentary}: As of February 2022, 7 of the participating
projects are closed. Out of these closed projects, 57\% provide their
data in the Epidoc XML format on their website, 43\% provide their data
via public repository on GitHub, 29\% via other publicly accessible
repositories, such as ILC4CLARIN Repository. 14\% of closed projects
don't currently share data outside the project (=1 project).

The fact that even the closed projects share their data in some form
even after their project is no longer active/does not have funding for
further development or maintenance is positive. However, most of the
data sit on private or institutional websites that can easily disappear,
along with access to the data. The best practice for the longevity of
the created datasets would be archiving them to a publicly accessible
repository, either GitHub, Zenodo, HAL, Open Science Framework or any
similar archival infrastructure.

\begin{center}\rule{0.5\linewidth}{0.5pt}\end{center}

\hypertarget{institutional-policies-1}{%
\subsection{Institutional policies}\label{institutional-policies-1}}

\emph{Question:} \emph{Does your institution or funding body require
your project to comply with any data policies (e.g., FAIR principles,
data storage, data sharing, Open Science)?}

To see more data, click on the arrow in the top right hand corner of the
table below.

\begin{longtable}[]{@{}
  >{\raggedright\arraybackslash}p{(\columnwidth - 4\tabcolsep) * \real{0.9471}}
  >{\raggedleft\arraybackslash}p{(\columnwidth - 4\tabcolsep) * \real{0.0176}}
  >{\raggedleft\arraybackslash}p{(\columnwidth - 4\tabcolsep) * \real{0.0353}}@{}}
\toprule
\begin{minipage}[b]{\linewidth}\raggedright
policies
\end{minipage} & \begin{minipage}[b]{\linewidth}\raggedleft
n
\end{minipage} & \begin{minipage}[b]{\linewidth}\raggedleft
ratio
\end{minipage} \\
\midrule
\endhead
No & 11 & 58 \\
Yes & 3 & 16 \\
Neither our grant funding (NEH), private funding, nor institutional
funding REQUIRES compliance with data policies, but all three encourage
open data practices. & 1 & 5 \\
Not with an official request, at the moment & 1 & 5 \\
Policies are on the way, but not yet established. & 1 & 5 \\
The French National Centre for Scientific Research strongly encourages
its members to comply with the FAIR principles. & 1 & 5 \\
We don't work for any institution & 1 & 5 \\
\bottomrule
\end{longtable}

\textbf{Commentary}: 58\% of projects do not explicitly have to follow
any policy. 16\% of projects are required to comply with data related
policies, while an additional 20\% of projects are encouraged to comply
with FAIR data principles but no rules are enforced.

\begin{center}\rule{0.5\linewidth}{0.5pt}\end{center}

\textbf{Question}: \emph{If you have answered YES in the previous
question, please specify what are the policies, or provide a link.}

\begin{verbatim}
## [1]
"https://www.uio.no/english/for-employees/support/research/research-data-management/fair-data/"
## [2] "All : French \"Plan national pour la science ouverte:Open Science\",
https://www.ouvrirlascience.fr/plan-national-pour-la-science-ouverte/; FAIR principles,
Mandatory deposit of our publications on the open archive HAL,
https://hal.archives-ouvertes.fr/"
## [3] "Creative Commons"
## [4] "data sharing"
\end{verbatim}

\textbf{Commentary}: Digital policies in the field of digital epigraphy
are still being implemented, which does not reflect yet on past and
current projects. There is a variation between national policies amongst
our responses, with France providing a vocal example in the
implementation of Open Science in digital epigraphy.

When we compare the average duration of the project and the requirement
to follow any institutional policies regarding the FAIR data, we see
that established projects are more likely required to follow such
policies than short and mid-term projects.

\begin{longtable}[]{@{}lr@{}}
\toprule
policy\_simple & average\_duration\_yr \\
\midrule
\endhead
N/A & 4.285714 \\
No & 10.071429 \\
Voluntary & 4.000000 \\
Yes & 35.750000 \\
\bottomrule
\end{longtable}

\textbf{Additional investigation}:

The main factor influencing the need to comply with institutional
principles seems to be however also the age of the project - for
projects created in recent years (e.g., since 2015), we would expect
FAIR data policy being one of conditions for securing funding. In order
to verify this hypothesis, we collected the additional information
manually from projects websites and published materials, such as the
official project start-date, the country of origin of a given project,
indication of existing funding and primary focus of a given project
(text publication or metadata collection); the anonymised data are saved
as a TSV in the same GitHub repository as
\texttt{/data/02\_scoping\_survey\_anonymised\_PostSurvey.tsv}.

\begin{figure}

{\centering \includegraphics{01_FAIR_epi_report_files/figure-latex/unnamed-chunk-36-1} 

}

\caption{The chart shows the existence of institutional data policy of non-partnered projects in time.}\label{fig:unnamed-chunk-36}
\end{figure}

Our expectations on data policies being progressively implemented over
the last seven years were confirmed only partially. As the chart below
demonstrates, the number of projects that indicated existing
institutional data policy grows steadily since 2015 (indicated by brown
dashed vertical line). On contrary, the number of projects that
indicated no existing data policy decreases, but only relatively slowly.
The projects responded \texttt{N/A} are those who consider themselves in
February 2022 as \texttt{closed}.

The following figure shows clear geographic differences in
implementation of data policies based on the main country where the
project is based.

\begin{figure}

{\centering \includegraphics{01_FAIR_epi_report_files/figure-latex/unnamed-chunk-38-1} 

}

\caption{The chart shows the requirement of institutional data policy of non-partnered projects by country.}\label{fig:unnamed-chunk-38}
\end{figure}

France, Germany, and Italy are listed as countries where data policies
are required, yet some projects in Germany and Italy answered that no
data policies are required. Thus the practical implementation of data
policies may depend on particular funding agency or institution, rather
than on nation-wide policies. However, our results are only preliminary
and based on a very small sample and need to be confirmed by further
investigation.

\hypertarget{open-science-practice-1}{%
\subsection{Open Science Practice}\label{open-science-practice-1}}

\textbf{Question}: \emph{Are you familiar with the FAIR data
principles?}

\begin{longtable}[]{@{}lrr@{}}
\toprule
policy & n & ratio \\
\midrule
\endhead
Yes & 19 & 73 \\
Vaguely & 6 & 23 \\
No & 1 & 4 \\
\bottomrule
\end{longtable}

\textbf{Commentary}: The majority of projects (73\%) is familiar with
FAIR data policy, however, 23\% of participating projects are familiar
only vaguely and would benefit from clear guidelines customised for the
epigraphic community. Only 4\% of projects are not familiar with FAIR
data principles.

\begin{center}\rule{0.5\linewidth}{0.5pt}\end{center}

\textbf{Question}: \emph{Standardized terminologies: The project uses
the following systems:}

To see more data, click on the arrow in the top right hand corner of the
table below.

\begin{longtable}[]{@{}
  >{\raggedright\arraybackslash}p{(\columnwidth - 4\tabcolsep) * \real{0.9494}}
  >{\raggedleft\arraybackslash}p{(\columnwidth - 4\tabcolsep) * \real{0.0169}}
  >{\raggedleft\arraybackslash}p{(\columnwidth - 4\tabcolsep) * \real{0.0337}}@{}}
\toprule
\begin{minipage}[b]{\linewidth}\raggedright
standard\_terminologies
\end{minipage} & \begin{minipage}[b]{\linewidth}\raggedleft
n
\end{minipage} & \begin{minipage}[b]{\linewidth}\raggedleft
ratio
\end{minipage} \\
\midrule
\endhead
Internal authority lists & 13 & 37 \\
EAGLE vocabularies as provided at
\url{https://www.eagle-network.eu/resources/vocabularies/} & 7 & 20 \\
Own version of EAGLE vocabularies (edited for our project) & 6 & 17 \\
We don't use any standardized lists & 3 & 9 \\
\url{https://epigraphie.mom.fr} & 1 & 3 \\
The project suggests the use of vocabularies in digital projects dealing
with ancient writing cultures & 1 & 3 \\
We created our own thesaurus with OpenTheso tool (EpiVoc)
\url{https://thesaurus.mom.fr/opentheso/?idt=th61} and we aligne with
existing vocabularies (work still in progress) & 1 & 3 \\
We generated a system for metadata based on the UBC library's ability to
categorize objects (it was very limited for ancient objects) & 1 & 3 \\
We use standard Mycenological terms but the community does not yet have
standardized lists. & 1 & 3 \\
We use the data provided by Konkordanz der Hethitischen
Keilschrifttafeln (www.hethiter.net/hetkonk) & 1 & 3 \\
\bottomrule
\end{longtable}

\textbf{Commentary}: 9\% of projects don't use any standardized lists or
vocabularies. 37\% of projects use their own internal authority lists.
EAGLE vocabularies in their original form are used by 20\% of projects,
and in an edited version by 17\% of projects. Several projects, that
focus on languages other than Greek and Latin, have created their own
systems, sometimes working from existing vocabularies, but also building
thesauri, e.g.~the response: We created our own thesaurus with OpenTheso
tool (EpiVoc) \url{https://thesaurus.mom.fr/opentheso/?idt=th61} and we
aligne with existing vocabularies (work still in progress) or We use
standard Mycenological terms but the community does not yet have
standardized lists..

\begin{center}\rule{0.5\linewidth}{0.5pt}\end{center}

\textbf{Question}: \emph{Are you willing to share the standardized
terminologies used in your project with us (e.g.~type of inscription
vocabularies, type of material etc.)}

\begin{longtable}[]{@{}lrr@{}}
\toprule
policy\_share & n & ratio \\
\midrule
\endhead
Yes & 23 & 88 \\
No & 3 & 12 \\
\bottomrule
\end{longtable}

\textbf{Commentary}: The vast majority of participating projects (88\%)
is willing to share any standardized terminologies used in their
project, such as terminologies covering the type of inscription
vocabularies, the type of material etc.

\begin{center}\rule{0.5\linewidth}{0.5pt}\end{center}

\textbf{Question}: \emph{Linked Open Datasets: The project uses the
following systems:}

To see more data, click on the arrow in the top right hand corner of the
table below.

\begin{longtable}[]{@{}
  >{\raggedright\arraybackslash}p{(\columnwidth - 4\tabcolsep) * \real{0.8696}}
  >{\raggedleft\arraybackslash}p{(\columnwidth - 4\tabcolsep) * \real{0.0435}}
  >{\raggedleft\arraybackslash}p{(\columnwidth - 4\tabcolsep) * \real{0.0870}}@{}}
\toprule
\begin{minipage}[b]{\linewidth}\raggedright
linked\_data
\end{minipage} & \begin{minipage}[b]{\linewidth}\raggedleft
n
\end{minipage} & \begin{minipage}[b]{\linewidth}\raggedleft
ratio
\end{minipage} \\
\midrule
\endhead
Pleiades & 13 & 50 \\
Trismegistos & 13 & 50 \\
EAGLE vocabularies & 9 & 35 \\
LGPN & 7 & 27 \\
None & 3 & 12 \\
PIR & 3 & 12 \\
diacritical marks from Leiden (CIL) & 1 & 4 \\
Geonames & 1 & 4 \\
GODOT: \url{https://godot.date/home} & 1 & 4 \\
I can't remember (sorry!) & 1 & 4 \\
iDaiGazetteer & 1 & 4 \\
idRef & 1 & 4 \\
None were yet available: a new edition will want to use all & 1 & 4 \\
Pactols & 1 & 4 \\
Period.O & 1 & 4 \\
PLRE & 1 & 4 \\
ToposTexts & 1 & 4 \\
under demand & 1 & 4 \\
We periodically ask to Trismegistos an ID for our records & 1 & 4 \\
\bottomrule
\end{longtable}

\textbf{Commentary}: Pleiades is the most popular LOD dataset, being
used in 50\% of all participating projects, as well as Trismegistos with
50\%. EAGLE vocabularies are represented in 35\% of participating
projects, while combined prosopographic datasets (LGPN+PIR) in 39\% of
projects. Only 12\% of participating projects do not use any LOD.

\hypertarget{future-needs-of-digital-epigraphy}{%
\section{Future needs of digital
epigraphy}\label{future-needs-of-digital-epigraphy}}

This section covers the wishes of partner projects as well as all
participating digital epigraphy projects. The responses were anonymised
so no individual or project can be identified but otherwise presented as
submitted in the survey.

\hypertarget{partner-projects}{%
\subsection{Partner projects}\label{partner-projects}}

\textbf{Question}: \emph{Our project would like to be able to use within
the next three years:}

To see more data, click on the arrow in the top right hand corner of the
table below.

\begin{longtable}[]{@{}
  >{\raggedright\arraybackslash}p{(\columnwidth - 4\tabcolsep) * \real{0.9259}}
  >{\raggedleft\arraybackslash}p{(\columnwidth - 4\tabcolsep) * \real{0.0123}}
  >{\raggedleft\arraybackslash}p{(\columnwidth - 4\tabcolsep) * \real{0.0617}}@{}}
\toprule
\begin{minipage}[b]{\linewidth}\raggedright
lod\_f
\end{minipage} & \begin{minipage}[b]{\linewidth}\raggedleft
n
\end{minipage} & \begin{minipage}[b]{\linewidth}\raggedleft
ratio\_all\_proj
\end{minipage} \\
\midrule
\endhead
Bibliographical references to all epigraphic publications with stable
URI (e.g.~Zenon) & 10 & 83 \\
EAGLE vocabularies (revised and extended with clear structure +
eliminated duplicates + multi-language support) & 9 & 75 \\
Roman Prosopographical data with stable URIs & 9 & 75 \\
Greek Onomastic data with stable URIs (e.g.~LGPN with stable
identifiers) & 6 & 50 \\
Open and accessible RDF Triplestore & 6 & 50 \\
One domain specific repository for epigraphic data & 5 & 42 \\
standardised terminologies for instrumentum domesticum and palaeography
& 1 & 8 \\
We are not sure what is meant by ``epigraphic data'' in the preceding
entry. If something like a papyri.info for inscriptions then no. If a
basic aggregator like Humanities Commons for epigraphy then that would
be nore useful. & 1 & 8 \\
\bottomrule
\end{longtable}

\textbf{Commentary}: The most popular is the option
\texttt{Bibliographical\ references\ to\ all\ epigraphic\ publications\ with\ stable\ URI\ (e.g.\ Zenon)}
represents the wishes of 83\% of all partner projects.

The great interest in prosopographic LOD for is supported by 75\% of
partner projects for the Roman world and 50\% of projects for the Greek
world respectively.

The \texttt{improved\ EAGLE\ vocabularies} are wished for by 75\% of
partner projects.

The \texttt{domain-specific\ repository\ for\ epigraphic\ data} or the
\texttt{open\ and\ accessible\ RDF\ Triplestore} do not seem to be the
highest priority of participating projects, but still relatively popular
as 42\% of responses wishes for one of the two. One participating
project wishes specifically for the following: standardised
terminologies for instrumentum domesticum and palaeography.

\begin{center}\rule{0.5\linewidth}{0.5pt}\end{center}

\textbf{Question}: \emph{Potential ideas that our project would benefit
from:}

\begin{longtable}[]{@{}
  >{\raggedright\arraybackslash}p{(\columnwidth - 4\tabcolsep) * \real{0.8022}}
  >{\raggedleft\arraybackslash}p{(\columnwidth - 4\tabcolsep) * \real{0.0330}}
  >{\raggedleft\arraybackslash}p{(\columnwidth - 4\tabcolsep) * \real{0.1648}}@{}}
\toprule
\begin{minipage}[b]{\linewidth}\raggedright
lod\_i
\end{minipage} & \begin{minipage}[b]{\linewidth}\raggedleft
n
\end{minipage} & \begin{minipage}[b]{\linewidth}\raggedleft
ratio\_all\_proj
\end{minipage} \\
\midrule
\endhead
Set of guidelines for FAIR and Linked Open Data in epigraphy & 11 &
92 \\
Practical scripted examples on how to use LOD in epigraphy & 9 & 75 \\
Workshop on FAIR principles in epigraphy & 7 & 58 \\
Set of guidelines/resources for quantitative analysis of epigraphic data
& 6 & 50 \\
Workshop on how to use LOD in epigraphy & 6 & 50 \\
NA & 1 & 8 \\
\bottomrule
\end{longtable}

\textbf{Commentary}: 92\% of all projects would benefit from
\texttt{A\ set\ of\ guidelines\ for\ FAIR\ and\ Linked\ Open\ Data\ in\ epigraphy}.
There is a general interest in practical examples and workshop(s) on how
to use LOD and FAIR Principles in Epigraphy, as well as resources for
quantitative analysis of data in epigraphy.

\begin{center}\rule{0.5\linewidth}{0.5pt}\end{center}

\textbf{Question}: \emph{Additional digital needs}

\begin{verbatim}
## [1] "Further development of a single research portal to interrogate multiple
epigraphic databases; development of a specific API to use the standardized common
vocabularies"
## [2] "- Further collaboration and development of concepts for vocabularies. - Getty
vocabularies crosswalks where they apply - In doing all this work, we hope that FAIR
Epigraphy will use as many different applications of the EpiDoc schema as possible, so as
to accommodate the ways different projects mark up documents and metadata."
## [3] "Sustainable common platform of all digital epigraphic editions (a Vision)"
## [4] "Advisory Board for new Digital Epigraphy projects, guidelines for FAIR epigraphy"
## [5] "Standards for palaeography, prosopography, bibliograpy, instrumentum domesticum
and linguistic analysis"
## [6] "1) Provide us support in switching to Epidoc XML encoding 2) Help us clarify how
our data are accessible to the public (CC-BY 4.0) 3) Promote the sustainability of
projects whose funds have ended 4) Foster interoperability between digital resources 5)
Improve the standardization of projects in digital epigraphy"
\end{verbatim}

\textbf{Commentary}: This section covers the additional needs of partner
projects. Partner projects would like to see a platform linking
epigraphic data from multiple sources, including a stable reference
point or an API for improved epigraphic vocabularies (in other words,
the sort of resource which agreed vocabularies and an RDF triplestore
might facilitate). Partner projects would also like to be able to use
guidelines for FAIR practices in epigraphy, which currently do not
exist.

\hypertarget{non-partnered-projects}{%
\subsection{Non-partnered projects}\label{non-partnered-projects}}

\textbf{Question}: \emph{Our project would like to be able to use within
the next three years:}

To see more data, click on the arrow in the top right hand corner of the
table below.

\begin{longtable}[]{@{}
  >{\raggedright\arraybackslash}p{(\columnwidth - 4\tabcolsep) * \real{0.8615}}
  >{\raggedleft\arraybackslash}p{(\columnwidth - 4\tabcolsep) * \real{0.0231}}
  >{\raggedleft\arraybackslash}p{(\columnwidth - 4\tabcolsep) * \real{0.1154}}@{}}
\toprule
\begin{minipage}[b]{\linewidth}\raggedright
lod\_f
\end{minipage} & \begin{minipage}[b]{\linewidth}\raggedleft
n
\end{minipage} & \begin{minipage}[b]{\linewidth}\raggedleft
ratio\_all\_proj
\end{minipage} \\
\midrule
\endhead
Bibliographical references to all epigraphic publications with stable
URI (e.g.~Zenon) & 17 & 65 \\
EAGLE vocabularies (revised and extended with clear structure +
eliminated duplicates + multi-language support) & 17 & 65 \\
Greek Onomastic data with stable URIs (e.g.~LGPN with stable
identifiers) & 13 & 50 \\
One domain specific repository for epigraphic data & 11 & 42 \\
Roman Prosopographical data with stable URIs & 11 & 42 \\
Open and accessible RDF Triplestore & 6 & 23 \\
None & 2 & 8 \\
Geolocation of inscriptions and searches related to geography & 1 & 4 \\
In the case of our project most of the options are not applicable & 1 &
4 \\
LGPN does not yet contain Mycenaean names but I would be happy if that
changed & 1 & 4 \\
This project is currently closed & 1 & 4 \\
\bottomrule
\end{longtable}

\textbf{Commentary}: The most popular is the option
\texttt{Bibliographical\ references\ to\ all\ epigraphic\ publications\ with\ stable\ URI\ (e.g.\ Zenon)}
representing the wishes of 65\% of all participating projects. The great
interest in onomastic and prosopographic LOD for both the Greek and
Roman world is supported by 92\% of non-partnered projects. The
\texttt{improved\ EAGLE\ vocabularies} are wished for by 65\% of
non-partnered projects. The
\texttt{domain-specific\ repository\ for\ epigraphic\ data} (23\%) or
the \texttt{open\ and\ accessible\ RDF\ Triplestore} (42\%) do not seem
to be the highest priority of participating projects, but still a
relatively popular response. One participating project wishes
specifically for the following: Geolocation of inscriptions and searches
related to geography, which other existing projects, such as Pleiades or
Trismegistos, might be better equipped to do.

\begin{center}\rule{0.5\linewidth}{0.5pt}\end{center}

\textbf{Question}: \emph{Potential ideas that our project would benefit
from:}

To see more data, click on the arrow in the top right hand corner of the
table below.

\begin{longtable}[]{@{}
  >{\raggedright\arraybackslash}p{(\columnwidth - 4\tabcolsep) * \real{0.8626}}
  >{\raggedleft\arraybackslash}p{(\columnwidth - 4\tabcolsep) * \real{0.0229}}
  >{\raggedleft\arraybackslash}p{(\columnwidth - 4\tabcolsep) * \real{0.1145}}@{}}
\toprule
\begin{minipage}[b]{\linewidth}\raggedright
lod\_i
\end{minipage} & \begin{minipage}[b]{\linewidth}\raggedleft
n
\end{minipage} & \begin{minipage}[b]{\linewidth}\raggedleft
ratio\_all\_proj
\end{minipage} \\
\midrule
\endhead
Set of guidelines for FAIR and Linked Open Data in epigraphy & 22 &
85 \\
Practical scripted examples on how to use LOD in epigraphy & 17 & 65 \\
Set of guidelines/resources for quantitative analysis of epigraphic data
& 17 & 65 \\
Workshop on how to use LOD in epigraphy & 16 & 62 \\
Workshop on FAIR principles in epigraphy & 11 & 42 \\
In the next three years we planned a few Digital Epigraphy workshops in
the frame of the French School at Athens & 1 & 4 \\
None & 1 & 4 \\
\bottomrule
\end{longtable}

\textbf{Commentary}: 85\% of all non-partnered projects would benefit
from
\texttt{A\ set\ of\ guidelines\ for\ FAIR\ and\ Linked\ Open\ Data\ in\ epigraphy}.
There is a general interest in practical examples (65\%) and workshop(s)
on how to use LOD in epigraphy (62\%), as well as resources for
quantitative analysis of data in epigraphy (62\%). There might be
potential synergy in organising workshops in digital epigraphy between
some of the participating projects,
e.g.~\texttt{the\ French\ School\ at\ Athens}.

\begin{center}\rule{0.5\linewidth}{0.5pt}\end{center}

\textbf{Question}: \emph{Additional digital needs}

\begin{verbatim}
## [1] "Digitalization of Roman Inscriptions for dissemination and research"
## [2] "A workshop on integrating Mycenaean data into epigraphy?"
## [3] "Data retrieval also on spatial base: for example: from maps of the single
archaeological sites and single complexes (as plans or 3d scans of catacombs and
churches...). Links with the existing geographical and georeferenced resources.
Controlled and shared vocabulary about palaeographical features; Storage, search and
analysis of the 'aberrant forms' (not to be 'corrected') for Late Latin and
Late/Byzantine Greek words (and names)."
## [4] "The most important for me would be 1/ to have a more complete view of real FAIR
epigraphic projects and 2/a sustainable \"common place\" where to find resources + tools
and help + let's call it an improved EAGLE + and more \"international\""
## [5] "It would be very nice (but I might be a bit biased!) if FAIR Epigrahy would like
to help develop EFES (EpiDoc Front-End Services). For example by helping to make the
existing RDF data export functionality really usable even by less experienced people."
## [6] "I would love to see it revitalized and improved with FAIR and Linked Open Data
guidelines and other resources."
## [7] "Unicode for Punic"
## [8] "help to act in a shared dedicated academical environment and help in spreading
our results"
## [9] "FAIR Epigraphy's team can help us by providing advice on specifical topics"
## [10] "It would be useful to have an Open Access database of images of inscriptions
that are free from Copyright limits."
\end{verbatim}

\textbf{Commentary}: This section covers additional needs of
participating digital projects. Some of the wishes might be beyond the
scope of the FAIR Epigraphy project but the responses provide valuable
guidance and hint to some of the challenges the epigraphic discipline
will be facing in the near future. The responses may inspire other
projects with similar needs to join forces and potentially develop the
solution together. The FAIR Epigraphy project may offer one channel to
explore and collaborate on the meeting of these needs in future.

\hypertarget{summary}{%
\section{Summary}\label{summary}}

The present report demonstrates a great variation in the epigraphic
discipline in 2022. Although the majority of participating projects
record inscriptions in Latin and Greek, we see a diverse array of
projects expanding beyond the traditional boundaries of the discipline.
The projects participating in the survey involve well-established
projects that have existed over several decades, regional or thematic
corpora, and more specialised, short-term PhD projects. We have observed
a clear distinction between on the one hand projects with a long
tradition and most importantly with relatively stable institutional
support, which have access to institutional repositories, policies and
IT services and other, small-scale projects with limited support and
access to resources and training, typified by to short-term projects on
a specific topic that may lack access to long-term institutional
support. One of the missions of the FAIR Epigraphy project is to support
projects with limited access to resources by providing accessible and
comprehensible training and guidelines for FAIR and Linked Open Data
principles in epigraphy.

The established projects mostly follow the FAIR principles, although to
a variable extent. The majority of established projects share their data
under a Creative Commons license in one or more widely accepted formats
(with Epidoc XML being the most popular format for all types of projects
irrespective of their status and longevity). In general, the more
established projects provide more access points to their data as well as
more data formats than the projects with less institutional support. The
use of standardized terminologies is still limited and project-specific,
mostly due to the lack of uniformly accepted standards. On contrary, the
adoption of Linked Open Datasets (LOD) and creation of links within the
epigraphic datasets with stable identifiers to those LOD sources seems
to be fairly advanced, especially in the case of well-established LOD
domains such as Pleiades or Trismegistos, and to a lesser extent the
EAGLE vocabularies.

The non-partnered projects follow the FAIR principles, but to a lesser
degree than the established projects. There are, however, some
short-time projects that fulfill or exceed the requirements for Open
Science, but as a general rule the compliance is lower than in the case
of established projects. The reason for lower compliance is most likely
a combination of only a short-term institutional support, limited access
to IT support, poor accessibility of guidelines and discipline specific
training.

As to the current and future needs of digital epigraphy, there is a
clear demand for more LOD, especially for bibliographical references to
standard epigraphic corpora, standardisation of discipline-specific
vocabularies (improved EAGLE vocabularies), and prosopographic LOD for
the ancient world, all supported by training and providing accessible
resources and sets of guidelines for FAIR and Open epigraphy. The need
for an accessible and open platform connecting and linking various
epigraphic resources through a single access point is generally
supported, building on the experience of the EAGLE Project, but now
exploiting the model of Linked Open Data.

\hypertarget{bibliography}{%
\section{Bibliography}\label{bibliography}}

\hypertarget{refs}{}
\begin{CSLReferences}{1}{0}
\leavevmode\vadjust pre{\hypertarget{ref-assael_restoring_2022}{}}%
Assael Y., Sommerschield T., Shillingford B., Bordbar M., Pavlopoulos
J., Chatzipanagiotou M., Androutsopoulos I., Prag J. \& Freitas N. de
(2022). Restoring and attributing ancient texts using deep neural
networks. Nature 603 (7900): 280--283.
\url{https://doi.org/10.1038/s41586-022-04448-z}.

\leavevmode\vadjust pre{\hypertarget{ref-bagnall_pleiades_2006}{}}%
Bagnall R., Talbert R.J.A., Bond S., Becker J., Elliott T., Gillies S.,
Horne R., McCormick M., Rabinowitz A., Turner B. \& Twele R. (2006).
Pleiades: {A} community-built gazetteer and graph of ancient places.
\url{http://pleiades.stoa.org}.

\leavevmode\vadjust pre{\hypertarget{ref-bruun_epigraphy_2015}{}}%
Elliott T. (2015). Epigraphy and {Digital} {Resources}. Vol. 1. Oxford
University Press.
\url{https://doi.org/10.1093/oxfordhb/9780195336467.013.005}.

\leavevmode\vadjust pre{\hypertarget{ref-geser_wp15_2016}{}}%
Geser G. (2016). {WP15} {Study}: {Towards} a {Web} of {Archaeological}
{Linked} {Open} {Data}. Ariadne.
\url{http://legacy.ariadne-infrastructure.eu/wp-content/uploads/2019/01/ARIADNE_archaeological_LOD_study_10-2016.pdf}.

\leavevmode\vadjust pre{\hypertarget{ref-hermankova_inscriptions_2021}{}}%
Heřmánková P., Kaše V. \& Sobotkova A. (2021). Inscriptions as data:
Digital epigraphy in macro-historical perspective. Journal of Digital
History 1. \url{https://journalofdigitalhistory.org/en/issue/jdh001}.

\leavevmode\vadjust pre{\hypertarget{ref-mullen_manual_2021}{}}%
Mullen A. \& Bowman A.K. (2021). Manual of {Roman} everyday writing
{Volume} 1 {Volume} 1.
\url{https://latinnow.files.wordpress.com/2021/11/latinnow-mullen-and-bowman-2021-mrew-scripts-and-texts-1.pdf}.

\leavevmode\vadjust pre{\hypertarget{ref-orlandi_digital_2021}{}}%
Orlandi S. (2021). Digital {Projects} in {Epigraphy}: {Research}
{Needs}, {Technical} {Possibilities}, and {Funding} {Problems}. In:
Velasquéz Soriano I. \& Espinosa Espinosa D. (eds.). Epigraphy in the
{Digital} {Age} : {Opportunities} and challenges in the {Recording},
{Analysis} and {Dissemination} of {Inscriptions}. Archaeopress, Oxford,
p. 1--8.

\leavevmode\vadjust pre{\hypertarget{ref-roueche_inscriptions_2022}{}}%
Roueche C. (2022). Inscriptions of {Roman} {Tripolitania} 2021. The
Society for Libyan Studies.

\leavevmode\vadjust pre{\hypertarget{ref-roueche_inscriptions_2020}{}}%
Roueche C., Reynolds J. \& Bodard G. (2020). Inscriptions of {Roman}
{Cyrenaica} (2020). The Society for Libyan Studies.

\leavevmode\vadjust pre{\hypertarget{ref-tupman_where_2021}{}}%
Tupman C. (2021). Where {Can} {Our} {Inscriptions} {Take} {Us}?
{Harnessing} the {Potential} of {Linked} {Open} {Data} for {Epigraphy}.
In: Velasquéz Soriano I. \& Espinosa Espinosa D. (eds.). Epigraphy in
the {Digital} {Age} : {Opportunities} and {Challenges} in the
{Recording}, {Analysis} and {Dissemination} of {Inscriptions}.
Archaeopress, Oxford, p. 115--128.

\leavevmode\vadjust pre{\hypertarget{ref-wilkinson_fair_2016}{}}%
Wilkinson M.D., Dumontier M., Aalbersberg Ij.J., Appleton G., Axton M.,
Baak A., Blomberg N., Boiten J.-W., Silva Santos L.B. da, Bourne P.E.,
Bouwman J., Brookes A.J., Clark T., Crosas M., Dillo I., Dumon O.,
Edmunds S., Evelo C.T., Finkers R., Gonzalez-Beltran A., Gray A.J.G.,
Groth P., Goble C., Grethe J.S., Heringa J., Hoen P.A.C. 't, Hooft R.,
Kuhn T., Kok R., Kok J., Lusher S.J., Martone M.E., Mons A., Packer
A.L., Persson B., Rocca-Serra P., Roos M., Schaik R. van, Sansone S.-A.,
Schultes E., Sengstag T., Slater T., Strawn G., Swertz M.A., Thompson
M., Lei J. van der, Mulligen E. van, Velterop J., Waagmeester A.,
Wittenburg P., Wolstencroft K., Zhao J. \& Mons B. (2016). The {FAIR}
{Guiding} {Principles} for scientific data management and stewardship.
Scientific Data 3 (1): 160018.
\url{https://doi.org/10.1038/sdata.2016.18}.

\leavevmode\vadjust pre{\hypertarget{ref-willi_manual_2021}{}}%
Willi A. (2021). Manual of {Roman} everyday writing {Volume} 2 {Volume}
2.
\url{https://latinnow.files.wordpress.com/2021/06/willi-2021-writing-equipment-latinnow.pdf}.

\end{CSLReferences}

\end{document}
